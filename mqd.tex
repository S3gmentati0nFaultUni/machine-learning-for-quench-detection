\documentclass[a4paper]{book}

\usepackage[parts]{classicthesis}

\title{\textsf{Supervised Machine Learning techniquest for quench detection in superconductors}}
\author{Alessandro Biagiotti}
\date{Academic Year 2023/2024}

\begin{document}

\frontmatter

\maketitle

\tableofcontents

\mainmatter

\chapter{An overview of superconductor quench}
Give a brief overview of what superconductor quench is.
\section{Superconductivity}
Give a brief overview of what superconductivity is at the microscopic level (London pairs,
interactions with the reticle, etc...)
\section{Type I superconductors}
Give a very rapid overview of what Type I superconductors are
\section{Type II superconductors}
Give insights about Type II superconductors
\chapter{An overview of the used machine learning models}
Chapter dedicated to a brief explanation of what machine learning is as well as the inner workings
of some of the used models.
\section{Supervised models}
An overview of what supervised models are
\subsection{DecisionTreeClassifier}
theoretical description of decision trees
\subsection{RandomForestClassifier}
theoretical description of random forests
\subsection{SVM}
theoretical description of SVMs
\section{Unsupervised models}
An overview of what unsupervised models are
\subsection{Clustering with k\-means}
An overview of k\-means clustering in theory
\chapter{The problem}
Section dedicated to talking about the problem
\section{Quench Recognition Problem (QRP)}
What do we expect from the Quench recognition problem? What are our labels?
\section{Quench Localization Problem (QLP)}
What do we expect from the Quench Localization problem? What are our labels?
\chapter{Solution for QRP}
A chapter entirely dedicated to the resolution of the QRP
\section{The data}
Description of the data and the labels.
\section{The models in Scikit\-learn?}
A brief description of the various models in Scikit\-learn
\section{TreeAggregator}
An overview of the Tree Aggregator structure
\section{Performance analysis}
Final performance analysis in which I show that among all of the possible models the best performing
seems to be the TA and I also show that they perform well in the blind test.
\chapter{Solution for QLP}
A chapter entirely dedicated to the resolution of the QLPo
\section{The data}
Description of the data and the labels.
\section{First approaches using Clustering}
\section{Extension of the QRP models to the new problem}
\section{Performance analysis}
\chapter{Further developmens and ideas}
\chapter{Conclusions}


%\addcontentsline{toc}{chapter}{List of Figures}
%\listoffigures
%\addcontentsline{toc}{chapter}{List of Tables}
%\listoftables
%
%\mainmatter
%
%\setcounter{chapter}{-1}
%\chapter{introduction}
This thesis was born as a cooperation between the department of Computer Science and Physics of
Milan University in an attempt of confirming or giving us a good reason to doubt the analytical
method originally found by Samuele Mariotto in his research on the localization of quench events for
the High Order Correctors superconducting magnets that will be mounted on the LHC experiment for the
HiLumi upgrade.

In the following I will be explaining some of the studies and results obtained in the field of
quench event recognition and quench event localization by using machine learning models to abstract
patterns from the measurements of magnetic field harmonics captured by Mariotto during his own
experiments.

In this document I chose to approach the matter as follows:
\begin{description}
	\item[\Cref{chp:soupcond-quench}] gives an overview of superconductivity and quench on a
		theoretical level.
	\item[\Cref{chp:ml}] gives an overview of the various models that have been used in the
		project on a theoretical level.
	\item[\Cref{chp:problem}] explains the challenges associated with the project, the datasets
		that have been used and the various testing and model selection procedures.
	\item[\Cref{chp:qrp}] introduces the Quench Recognition Problem, as well as the models
		utilized to solve the problem and the results obtained.
	\item[\Cref{chp:qlp}] introduces the Qench Localization Problem, the extension of previously
		obtained models to this new instance, and the final results.
	\item[\Cref{chp:future}] lists other ideas that came along while designing a solution for
		the two problems at hand.
	\item[\Cref{chp:conclusion}] closes the document with a couple of thoughts on the design and
		development process.
\end{description}

%\include{src/context}
%\include{src/preliminaries}
%
%\part{On sweeping \texorpdfstring{$k$}{k}-limited automata}
%\include{src/techniques}
%\include{src/sweeplimited}
%\include{src/nondeterminism}
%
%\part{Other models}
%\include{src/unaryoncemarking}
%
%\appendix
%\cleardoublepage
%\part*{Appendix}
%\include{src/complresults}
%\include{src/constructions}
%
%\hfuzz=2pt
%\tolerance=257
%\printbibliography

\end{document}
