\documentclass[a4paper]{book}

\usepackage[parts]{classicthesis}

\title{\textsf{Supervised Machine Learning techniques for quench detection in superconductors}}
\author{Alessandro Biagiotti}
\date{Academic Year 2023/2024}

\usepackage{mqd}

\begin{document}

\frontmatter

\maketitle

\tableofcontents

\mainmatter

\chapter{Superconductivity and quench: an overview}
In this chapter I will explain some of the very basic concepts linked to superconductors and
superconductor quench.
\section{Superconductivity in theory}
Electrical motion in metallic or insulating materials is a very well studied and documented
phenomenon. When the first evidences of superconductivity were found in 1911
\cite{invention-superconductivity} it soon was discovered that
the underlying theory was not actually able to explain the newfound properties.
Let's consider a cable of length $l$ and cross-section $A$ exerting a resistance $R$, the material
has an associated value of resistivity $\rho$ which can be computed as follows
\begin{equation}
	\label{eq:resistivity-cable}
	\rho = \frac{RA}{l} = \frac{m}{ne^2\tau}
\end{equation}
because of the second part of \ref{eq:resistivity-cable} resistivity can also be defined as in dependence of the mass of the electron $m$, the charge
density $n$, the electrical charge of the electron $e$ and the relaxation time of the electron, which indicates the time interval occurring between two successive collisions between
electrons inside a conductor, $\tau$.

The resistivity of any conductor can also be written in terms of the sample's temperature and a reference
value of resistivity ($\rho_0$) and temperature ($T_0$)
\begin{equation}
	\label{eq:resistivity-func-of-temp}
	\rho_T = \rho_0[1 + \alpha(T - T_0)]
\end{equation}
$\alpha$ is a regularization term known as the temperature coefficient of resistivity, which is the
ratio of resistance change to temperature change.

As it's explained in \cite{slimani2022superconducting} the result of fusing solid band theory with
\ref{eq:resistivity-func-of-temp} behavior of metals and insulators can be explained in terms
of the freedom of movement of electrons in the crystal. An increase in temperature for a metal
translates into an increase in resistivity while insulators experience a decrease in resistivity
and some start behaving like metals.
\section{Type I superconductors}
\section{Type II superconductors}
\section{Quench in theory}

\chapter{Machine learning: an overview}
\section{Supervised models}
\subsection{Decision Trees}
\subsection{Random Forests}
\subsection{SVM}
\section{Unsupervised models}
\subsection{Clustering with k\-means}
\chapter{The problem}
\section{The challenges}
\section{The datasets and their meaning}
\section{Model selection and model testing procedures}
\chapter{Quench Recognition Problem (QRP)}
\section{Problem description}
\section{Results}
\chapter{Quench Localization Problem (QLP)}
\section{Problem description}
\section{First approaches using Clustering}
\section{Extension of the QRP models}
\section{Results}
\chapter{Further developments and ideas}
\chapter{Conclusions}

\printbibliography

%\addcontentsline{toc}{chapter}{List of Figures}
%\listoffigures
%\addcontentsline{toc}{chapter}{List of Tables}
%\listoftables
%
%\mainmatter
%
%\setcounter{chapter}{-1}
%\chapter{introduction}
This thesis was born as a cooperation between the department of Computer Science and Physics of
Milan University in an attempt of confirming or giving us a good reason to doubt the analytical
method originally found by Samuele Mariotto in his research on the localization of quench events for
the High Order Correctors superconducting magnets that will be mounted on the LHC experiment for the
HiLumi upgrade.

In the following I will be explaining some of the studies and results obtained in the field of
quench event recognition and quench event localization by using machine learning models to abstract
patterns from the measurements of magnetic field harmonics captured by Mariotto during his own
experiments.

In this document I chose to approach the matter as follows:
\begin{description}
	\item[\Cref{chp:soupcond-quench}] gives an overview of superconductivity and quench on a
		theoretical level.
	\item[\Cref{chp:ml}] gives an overview of the various models that have been used in the
		project on a theoretical level.
	\item[\Cref{chp:problem}] explains the challenges associated with the project, the datasets
		that have been used and the various testing and model selection procedures.
	\item[\Cref{chp:qrp}] introduces the Quench Recognition Problem, as well as the models
		utilized to solve the problem and the results obtained.
	\item[\Cref{chp:qlp}] introduces the Qench Localization Problem, the extension of previously
		obtained models to this new instance, and the final results.
	\item[\Cref{chp:future}] lists other ideas that came along while designing a solution for
		the two problems at hand.
	\item[\Cref{chp:conclusion}] closes the document with a couple of thoughts on the design and
		development process.
\end{description}

%\include{src/context}
%\include{src/preliminaries}
%
%\part{On sweeping \texorpdfstring{$k$}{k}-limited automata}
%\include{src/techniques}
%\include{src/sweeplimited}
%\include{src/nondeterminism}
%
%\part{Other models}
%\include{src/unaryoncemarking}
%
%\appendix
%\cleardoublepage
%\part*{Appendix}
%\include{src/complresults}
%\include{src/constructions}
%
%\hfuzz=2pt
%\tolerance=257

\end{document}
