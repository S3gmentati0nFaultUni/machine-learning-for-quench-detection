\documentclass[a4paper, notitlepage]{article}

\usepackage{src/summary}

\title{
	\textsf{Supervised Machine Learning techniques for quench detection in superconductors} \\
	\author{Alessandro Biagiotti -- 13903A}
        \date{Febraury 20, 2025}
}

\begin{document}
\maketitle
\section{Hosting laboratory}
This thesis originated as a cooperation between the departments of Computer Science and Physics of
Milan University. Three were the advisors for the work:
\begin{itemize}
	\item Dario Malchiodi, associate professor of the Computer science department of Milan University;
	\item Samuele Mariotto, researcher for \textsc{infn} and Physics department of Milan
	      University;
	\item Lucio Rossi;
\end{itemize}
\todo{Sistemare la parte dei ruoli (ammesso e non concesso che serva) -- Ale}
\section{Initial context}
Our work attempted to confirm the analytical method, originally found by \textsc{infn} and the
University of Milan superconducting research group, for the localization of quench events in High
Order Corrector superconducting magnets \cite{mariotto2022-hoc, mariotto2022-generic} that will be
mounted in the \textsc{lhc} collider machine for the High Luminosity upgrade \cite{rossi2024-hllhc}
(\textsc{hl-lhc}).

The aim of the original work was, beginning from the harmonic decomposition of the magnetic
field produced by the superconducting magnet after a \emph{quench} event, which is a transition of the
material from the superconducting state to the normal conducting state, to find an analytical
explanation for the reconstructed harmonic content and correlate the information with the
localization of the quenched superconducting coil(s) in the magnet assembly. This is
referred to as the Quench Localization Problem (\qlp\ for short).

The magnetic field was measured at the center of the magnet, where the beam will pass during
production, and is the result of the contribution of the magnetization of the different coils and of
the iron yoke that locks the components in place and avoids their drift due to Lorentz forces.

\section{Objectives of the work}
Given the harmonic decomposition of magnetic field measured on the magnet during normal operation
and after quench events, the aim of this thesis was to identify two models: one capable of reliably
replying to the question 'Did the magnet quench?', and one capable of reliably replying to the
question 'If quench happened during the test run, which coil(s) within the magnet did quench?'.

Our objective was to find models with high performance levels and high explainability since the
original idea was to use the results for different studies:
\begin{itemize}
	\item understand which magnetic field harmonics are more important for the analysis of
	      quench events in superconducting magnets,
	\item using the results of the analysis to do better instrument calibration for Quench
	      Protection Systems and quench antennae.
\end{itemize}

\section{Description of the work done}
Due to the nature of the problem and the explainability requirements we mainly focused on machine
learning models that are inherently explainable.

During the internship we solved three different problems:
\begin{itemize}
	\item Quench Recognition Problem (\qrp), centered on the process of detecting quench events in
	      quadrupolar magnets;
	\item Quench Localization Problem (\qlp), centered on the process of localizing a quench event;
	\item Quench Recognition Problem for a specific coil (\qrpi), centered on the process of detecting
	      quench events in specific coils.
\end{itemize}

\section{The technologies we used}
Since the dataset available to us was already labelled by my co-advisor we mainly used supervised
machine learning models that have a strong inherent explainability. The models we used were the
following:
\begin{itemize}
	\item Decision Trees (\dts)~\cite{breiman1984-dt}, extremely explainable, versatile and simple models. We chose them as a
	      starting point;
	\item Random Forests (\rfs)~\cite{breiman2001-rf}, an ensemble based on trees, explainable by definition. We chose them
	      as an upgrade over (\dts) thanks to the higher performance;
	\item Tree aggregators (\tas), a custom ensemble based on \dts\ we chose specifically for
	      their performance;
	\item Support Vector Machines (\svms)~\cite{cortes1995-svm}, strong performers without the
	      explainability property that we considered the model to beat.
\end{itemize}
We worked on the project using Python and scikit-learn.

\section{Results}
I approached the project as a challenge against myself, since the in the beginning my machine
learning knowledge was very limited. At the end of the experience I can say that I learned how to
approach a machine learning project, as well as some a bit of the field of superconductor physics.
As an added bonus I trained soft-skills like: Designing presentations, exposing results to
colleagues with minimal experience in the field of machine learning and computer science.

At the moment of writing we obtained a model capable of finding a solution for \qrp\ with a very
high level of performance, we also got some extremely promising initial results for \qlp. While
developing the project we had various ideas to further explore the field:
\begin{enumerate}
	\item Have a physician study the decision rules to understand whether they have physical
	      foundation,
	\item fuzzification of the quench-event description; to explore other aspects of the event
	      like the velocity of propagation or the violence of the strength of the precursor,
	\item analysis of the voltage data to gain a different interpretation of the quench event,
	\item high-performance quench localization through neural networks or \svcs,
	\item clustering-based approaces as a secondary classification strategy for \qlp.
\end{enumerate}

\printbibliography

\end{document}
