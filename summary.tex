\documentclass[a4paper, notitlepage]{article}

\usepackage{src/summary}

\title{
	\textsf{Supervised Machine Learning techniques for quench detection in superconductors} \\
	\author{Alessandro Biagiotti -- 13903A}
        \date{Febraury 20, 2025}
}

\begin{document}
\maketitle
\section{Hosting laboratory}
This thesis originated as a cooperation between the departments of Computer Science and Physics of
Milan University. We worked essentially at: 'Laboratorio Acceleratori e Superconduttività Applicata'
(\textsc{lasa}) and 'Laboratorio di Algoritmica per il Web' (\textsc{law}); my advisors for the
thesis have been:
\begin{itemize}
	\item Dario Malchiodi, associate professor of the Computer science department of Milan University;
	\item Samuele Mariotto, researcher for \textsc{infn} and Physics department of Milan
	      University;
	\item Lucio Rossi, associate professor of the Physics department of Milan University and
	      researcher for \textsc{infn}.
\end{itemize}
\todo{Sistemare la parte dei ruoli (ammesso e non concesso che serva) -- Ale}
\section{Initial context}
Our work attempted to confirm the analytical method, originally found by \textsc{infn} and University of Milan superconducting research group, for the localization of quench events in High Order Corrector superconducting magnets~\cite{mariotto2022-hoc, mariotto2022-generic} that will be mounted in the \textsc{lhc} collider machine for the High Luminosity upgrade~\cite{rossi2024-hllhc} (\textsc{hl-lhc}).

A quench event is a transition of a superconducting material from the superconducting state to the
normal-conducting state. This transition is accompanied by a rise in resistive voltage which can
lead to material damage due to Joule-heating.

Given the harmonic decomposition of magnetic field after a quench event; the goal of the original
work was to find an analytical description for the reconstructed harmonic content and to correlate
the information with the localization of the quenched superconducting coil(s) in the magnet
assembly. The magnetic field was measured at the center of the magnet, where the beam will pass
during production, and is the result of the contribution of the magnetization of the different coils
and of the iron yoke that locks the components in place and avoids their drift due to Lorentz
forces.

\section{Objectives of the work}
Given the harmonic decomposition of magnetic field measured on the magnet during normal operation
and after quench events; the aim of this thesis was to identify two models: one capable of reliably
replying to the question 'Did the magnet quench?' (we referred to this problem as \textsc{qrp}), and
one capable of reliably replying to the question 'If quench happened during the test run, which
coil(s) within the magnet did quench?' (we referred to this problem as \textsc{qlp}).

It was of strong interest to find models capable of both yielding good performance while also being
easily explainable. Ideally we would expect to be able to:
\begin{itemize}
	\item understand which magnetic field harmonics are more important for the analysis of
	      quench events in superconducting magnets,
	\item use the results of the analysis to do better instrument calibration for Quench
	      Protection Systems and quench antennae.
\end{itemize}

\section{Description of the work done}
Since the measurements had already been made the thesis consisted mostly of analyzing the available
data under different lights and evaluating different models and their performance. Our data
contained only the measurements done exclusively on quadrupolar magnets because they are the ones
that showed the more inconsistent behavior across the original test campaign.

%During the internship we:
%\begin{itemize}
%	\item Found a strong performer capable of solving \qrp\ consistently;
%	\item We defined Quench Recognition Problem for coil $i$ (\qrpi), and found good and simple
%	      models capable of solving the problem for $i \in \{0, 1, 2, 3\}$;
%	\item Obtained very good initial results for the problem of \qlp\ by transforming it in the
%	      iterative resolution of \qrpi.
%\end{itemize}

\section{The technologies we used}
Since the dataset available to us had already been labelled, we mainly used supervised machine
learning models that have a strong inherent explainability. The models we used were the following:
\begin{itemize}
	\item Decision Trees (\dts)~\cite{breiman1984-dt}, extremely explainable, versatile and
	      simple models. Chosen as our main starting point;
	\item Random Forests (\rfs)~\cite{breiman2001-rf}, an ensemble based on trees, explainable by definition. We chose them as an upgrade over (\dts) thanks to the higher performance;
	\item Tree aggregators (\tas), a custom ensemble based on \dts; we designed them with
	      performance and simplicity in mind. Our objective with them was to reach performance
	      comparable to \rfs\ while having a simpler structure at the same time;
	\item Support Vector Machines (\svms)~\cite{cortes1995-svm}, strong performers without the
	      explainability property that we considered the model to beat.
\end{itemize}
For our study we chose python and the classic data science libraries, namely: pandas,
scikit-learn and numpy.

\section{Results}
I approached the project as a challenge, since my initial machine learning background was very
limited. At the end of the experience I learned: how to approach and maintain a machine learning
project, as well as some theoretical foundations of the field of superconductor physics. As an added
bonus I trained soft-skills by presenting my results and discussing the development of the project
with my advisors.

At the end of February we also sent our paper 'Quench detection and localization via
interpretable machine learning' to the Engineering Applications of Neural Networks conference
(approval pending).

At the time of writing we obtained a model capable of finding a solution for \qrp\ with a very
high level of performance, we also got some extremely promising initial results for \qlp. During
development, we had various ideas to further explore the field:
\begin{enumerate}
	\item Have a physicist study the decision rules of our models to understand if there is
	      any physical reason behind them,
	\item fuzzification of the quench-event description; to explore other aspects of the event
	      like the velocity of propagation or the strength of the precursor related to the
	      Quench Protection System activation,
	\item analysis of the voltage data to reach a different interpretation of the quench event,
	\item high-performance quench localization through neural networks or \svcs,
	\item clustering-based approaches as a secondary classification strategy for \qlp.
\end{enumerate}

\printbibliography

\end{document}
