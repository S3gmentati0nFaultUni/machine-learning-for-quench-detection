\documentclass[a4paper, notitlepage]{article}

\usepackage{src/summary}

\title{
	\textsf{Supervised Machine Learning techniques for quench detection in superconductors} \\
	\author{Alessandro Biagiotti -- 13903A}
        \date{Febraury 20, 2025}
}

\begin{document}
\maketitle
\section{Hosting laboratory}
This thesis originated as a cooperation between the departments of Computer Science and Physics of
Milan University. We worked essentially at: 'Laboratorio Acceleratori e Superconduttività Applicata'
(\textsc{lasa}) and 'Laboratorio di Algoritmica per il Web' (\textsc{law}); my advisors for the
thesis have been:
\begin{itemize}
	\item Dario Malchiodi, associate professor of the Computer science department of Milan University;
	\item Samuele Mariotto, researcher for \textsc{infn} and Physics department of Milan
	      University;
	\item Lucio Rossi, associate professor of the Physics department of Milan University and
	      researcher for \textsc{infn}.
\end{itemize}
\section{Initial context}
Our work attempted to confirm the analytical method, originally found by \textsc{infn} and
University of Milan superconducting research group, for the localization of quench events in High
Order Corrector superconducting magnets~\cite{mariotto2022-hoc, mariotto2022-generic} that will be
mounted in the Large Hadron Collider particle accelerator for the High Luminosity upgrade~\cite{rossi2024-hllhc} (\textsc{hl-lhc}).

A quench event is the transition of a superconducting sample to the normal-conducting state.
Leading to a buildup of resistive voltage, potentially causing material damage due to Joule-heating.

Given the harmonic decomposition of magnetic field produced by a superconducting magnet after a quench event, the goal of the original
work was to find an analytical description of the reconstructed harmonic content; and to correlate
the information with the localization of the quenched superconducting coil(s) in the magnet
assembly. This analysis aimed at predicting degradation of the superconducting material and
providing feedback on the quality and performances of the superconducting magnet. The magnetic field
was measured at the center of the magnet, where the beam will pass during acceleration and
collision. The measured field is the result of the contribution of the magnetization of the
different coils and of the iron yoke used in the magnet assembly to enhance the coil produced
magnetic field and shield the electromagnetic field out of the accelerator envelope.

\section{Objectives of the work}
As was said in the previous section, the data available to us consisted in the harmonic
decomposition of magnetic field measured on the magnet during normal operation or after quench
events, the aim of this thesis was to identify two models: one capable of reliably
replying to the question 'Did the magnet quench?' (we referred to this problem as \textsc{qrp}), and
one capable of reliably replying to the question 'If quench happened during the test run, which
coil(s) within the magnet did quench?' (we referred to this problem as \textsc{qlp}).

It was of strong interest to find models capable of both yielding good performance while also being
easily explainable. Ideally we would expect to be able to:
\begin{itemize}
	\item understand which magnetic field harmonics are more important for the analysis of
	      quench events in superconducting magnets,
	\item use the results of the analysis to develop better instrumentation and detection
	      systems, able to protect the magnet in real time and more reliably compared to
	      present quench protection methods used for superconducting magnets in particle
	      accelerators for high energy physics.
\end{itemize}

\section{Description of the work done}
Since the measurements had already been made the thesis consisted mostly of analysing different
aspects of the data and evaluating the performance of various supervised machine learning models. Our analysis
focused mainly on one type (quadrupole) of superconducting magnet developed in the High Order
Corrector magnet project. These magnets were designed to correct different types of multipolar
anomalies (quadrupolar, sextupolar, octupolar, decapolar, dodecapolar) introduced in the beam dynamics by
the focusing quadrupoles in proximity of the interaction regions.

These anomalies, if left unchecked, can give rise to collective effects like thermalization that
result in beam loss.

\section{The technologies we used}
The machine learning models we chose, shown below, for the task were characterized by a strong
interpretability, which made them suitable for later analyses.
\begin{itemize}
	\item Decision trees (\dts)~\cite{breiman1984-dt}, extremely explainable, versatile and
	      simple models. Chosen as our main starting point.
	\item Random forests (\rfs)~\cite{breiman2001-rf}, an ensemble based on trees, explainable
	      by definition. We chose them as an upgrade over (\dts) thanks to the higher
	      performance.
	\item Tree aggregators (\tas), a custom ensemble based on \dts; we designed them with
	      performance and simplicity in mind. Our objective with them was to reach performance
	      comparable to \rfs\ while having a simpler structure at the same time.
\end{itemize}
Our benchmark was the Support Vector Machine (\svm)~\cite{cortes1995-svm}, chosen because the model
yields very high performance, and represents the ideal results if we decided to rule-out the explainability constraint.

For our study we chose python and classic data science libraries, namely: pandas,
scikit-learn and numpy.

\section{Results}
At the end of February we also sent our paper 'Quench detection and localization via
interpretable machine learning' to the 'Engineering Applications and Advances of Artificial
Intelligence' conference and was recently accepted.

\medskip

At the time of writing we obtained a model capable of finding a solution for \qrp\ with a very
high level of performance, we also got some extremely promising initial results for \qlp. During
development, we had various ideas to further explore the field:
\begin{enumerate}
	\item have a physicist study the decision rules of our models, to correlate the model
	      results with a physical explanation of the produced magnetic field harmonic content.
	      Like performed in the analytical model aforementioned~\cite{mariotto2022-generic}.
	\item fuzzification of the quench event description; to explore other aspects of quench
	      events like the velocity of propagation or the precursor strength related to the Quench Protection System activation,
	\item analysis of the voltage data to explore a different interpretation of the quench event,
	\item high-performance quench localization through neural networks or \svcs, these models
	      would yield better performance and could be possibly used in a context of online
	      quench detection. But they would lack the explainability of the models identified in
	      the work here presented.
	\item clustering-based approaches as a secondary classification strategy for \qlp.
\end{enumerate}

\printbibliography

\end{document}
