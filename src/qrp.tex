\chapter{Quench Recognition Problem (QRP)}
\label{chp:qrp}
This chapter is dedicated to how we handled the first of the two problems that we introduced in the
perevious chapter, since $\qlp$ is an extension of $\qrp$ we began with the analysis of the easier
problem, all of the conclusions that we reach in this chapter, as far as the interaction between the
various harmonics is concerned, can be extended to the $\qlp$ problem as well.

In this chapter we will:
\begin{itemize}
	\item Give a rapid overview of the problem,
	\item Talk about the models used and the results obtained for each model,
	\item Select the 'best' model among the possibilities and then we will back our decision.
\end{itemize}

\section{Problem description}
As was introduced in the previous chapter we were given by Samuele Mariotto a dataset containing
$279$ samples divided between 'quench' and 'non-quench' events, the distribution is not exactly
balanced: $192$ quench events and $87$ non-quench events. The imbalance is noticeable but it's not
extreme, therefore we kept it in mind but without worrying about it too much.

Each sample is characterized by $15$ magnetic field harmonics, depending on the table we consider
the meaning of the single value is going to be different.

\section{Data preprocessing}

\section{Results}
