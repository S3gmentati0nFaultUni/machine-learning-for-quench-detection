\chapter*{Acknowledgements}
Every once in a while I like to look back at what I have achieved in the last three years, I
explored different fields, made great friends and lived in a different country. All of this is
thanks to the choice of staying in university.

If I had chosen to stop studying, as I was inclined on doing in the first place, I would definitely
have more money and I wouldn't be the main character of the new abc tv-series 'The walking Debt'
(patent pending). As per usual in these cases I would like to thank a bunch of people for the
inspiration and support they gave in the prior years and, especially, in the final months of this
master degree.

I want to thank Dario and Samuele, because they followed me on this path, and helped me take this
project to the end. I hope that our cooperation can keep on living for the forseable future and (why
not) for years to come.

I want to thank my family, especially my parents and my sister~\footnote{
	We are going climbing sooner or later, and that is a promise.
}, I want to thank \emph{all} of my friends, from the guys at Law, to i Russi, Vongoliani, Shibashi, El
Flachin, Lo Slavo, Alin, the list goes on and on. Special thanks go, of course, to Silke, who was
there to listen and to help, despite the distance, in the more difficult moments of the past months
(she also listened by her own choice, to the expanded version of the presentation, for that alone,
she deserves a medal).

I also want to dedicate this thesis to the friends who are actually working on their own at the
moment, or are going to start soon, especially: Francesco, who chose an amazing thesis full of
complexity and that can teach him so much, Noemi, who is one of the bravest people I know, Lucia \&
Coso, who are both doing incredible researches, and I am really looking forward to seeing what they can
come up with, and Edoardo, as a hope for the brilliant future waiting for him in America~\footnote{
	I hope you have fun doing your thesis in HYDROGEN
}~\footnote{
	https://www.youtube.com/watch?v=St7ny38gLp4
}.

To close these acknowledgements I want to answer a question: 'Why?'

I could have easily done a thesis in computer science and call it a day, without having to venture
in the world of particle physics (Heidelberg University) or accelerator physics (here in Unimi). The
truth is simply that I can't stand doing computer science for computer science's sake. Last year I
was in a bit of a dilemma because I had no clue of what I wanted to do. After all, a thesis is
non-binding, but it definitely shapes future choices.

In Germany I chose to follow courses that bridged the gap between computer science and physics. That
is when I discovered a hidden world, full of beauty and complexity, that we are currently trying to
concurrently tame and explore. Is there anything more elegant than spending billions of dollars,
and years in research, development and construction to have: a 27km ring hidden from sight, where two
counter-rotating hadronic beams are accelerated until their energy comes shy of 15 TeV, so that
they can reunite in a single point of interaction, where magnetic forces squeeze the particles as
close together as current technologies allow us; and some of the most advanced machines ever built
can illustrate the majestic universe painted by their final encounter.

That is true poetry in motion.
