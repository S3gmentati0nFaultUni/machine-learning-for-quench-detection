\chapter*{Acknowledgements}
At the end of my bachelor degree I was half thinking of ending my studies then and there, the idea
of making some money for myself was the main reason moving me in this direction. Back then, thanks
to my old advisor, Professor Zandron, I chose to start my master degree. Now, after three years, I
have explored different fields, discovered a potential path for my future and I lived in another
country for over $6$ months. If I had chosen to stop studying, as I was inclined on doing in the
first place, I would definitely have more money, and I wouldn't be the main character of the new abc
tv-series 'The walking Debt' (patent pending).

\medskip

As per usual, in these cases, I would like to thank a bunch of people for the inspiration and support
they gave in the last couple of years and in the final months of this master degree.

\smallskip

I want to thank Dario and Samuele, because they followed me on this path, and helped me take this
project to the end. I hope that our cooperation can keep on living for the forseable future and (why
not) for years to come.

\smallskip

I want to thank my family, especially my parents and my sister~\footnote{
	We are going climbing sooner or later, and that is a promise.
}, and all of my friends, from the guys at Law, to 'i Russi', 'i Vongoliani', Shibashi, El
Flachin, Lo Slavo, Alin, Giulia, the list goes on and on. Special thanks go, of course, to Silke, who was
there to listen and to help, despite the distance (she also listened, by her own choice, to the
expanded version of the presentation, for that alone, she deserves a medal).

\smallskip

\begin{minipage}{0.3\textwidth}
	\includegraphics[scale=0.4]{img/kek.png}
\end{minipage}%
\hfill
\begin{minipage}{0.65\textwidth}
	I also want to dedicate this thesis to the friends who are actually working on their own at the
	moment, or are going to start soon, especially: Francesco, who chose an amazing project full of
	complexity and that can teach him so much, Noemi, who bravely chose to change her project to move to
	something that she is truly fascinated by, Lucia \& Coso, who are both doing incredible
	researches, and I am really looking forward to seeing what they can come up with, and
	Edoardo, as a hope for the brilliant future waiting for him in America.
\end{minipage}

\bigskip

To close these acknowledgements I want to answer a question: 'Why?'

\medskip

In Germany I chose to follow courses that bridged the gap between computer science and physics (the
reason why is still unclear to me to this day). That is when I discovered a hidden world, full of
beauty and complexity, that we are trying to explore and understand. Is there anything more elegant
than spending billions of dollars, and years in research, development and construction to have: a
27km ring hidden from sight, where two counter-rotating hadronic beams are accelerated until their
energy comes shy of 15 TeV, so that they can reunite in a single point of interaction, where
magnetic forces squeeze the particles as close together as current technologies allow us; and some
of the most advanced machines ever built can illustrate the majestic universe painted by their final
interaction.

\bigskip

That is true poetry in motion.
