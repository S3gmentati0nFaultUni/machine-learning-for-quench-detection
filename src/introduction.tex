\chapter{Introduction}
This thesis originated as a cooperation between the departments of Computer Science and Physics of
Milan University in an attempt of confirming the analytical method, originally found by
\textsc{infn} and the University of Milan superconducting research group, for the localization of
quench events in High Order Corrector superconducting magnets \cite{mariotto2022,
	mariotto2022-generic} that will be mounted in the LHC collider machine for the High
Luminosity upgrade \cite{rossi2024} (\textsc{hl-lhc}).

The aim of the original work was to use the harmonic decomposition of the residual magnetic field of
a superconducting magnet, after a quench event; to find an analytical explanation for the
reconstructed harmonic content, and to correlate it with the localization of the quenched
superconducting coil(s) in the magnet assembly. A quench event is a situation in which a
superconductor transitions from the superconducting to the normal-conducting state; generating a
buildup of resistance that can lead to material damage due to Joule heating.

The aim of this thesis is to identify a machine learning model capable of answering the question: 'If quench
happened during the test run, which coil within the magnet did quench?'; knowing the harmonic
decomposition of the magnetic field measured at the center of the magnet, where the beam will pass
during production. This magnetic field is the result of the contribution of the magnetization of the
superconducting coils and the iron yoke that locks the coils in place. The analyses have been
conducted after a quench event; to avoid damaging the material the magnet was tested in a protected environment, connected to a Quench Protection System, (the experiment will be explained further in \Cref{chp:problem}).

In the following, we will be explaining some of the studies and results obtained in the field of
quench event recognition and quench event localization, using machine learning models to abstract
patterns from the measurements of magnetic field harmonics captured by the co-advisor for this
thesis, researcher Samuele, during his own experiments. The resulting models could be of use in the
field of magnet testing and behavior explanation for future research work in the field of magnet
manufacturing and magnet protection systems.

In this document we will explore the matter as follows:
\begin{description}
	\item[\Cref{chp:soupcond-quench}] gives an overview of superconductivity and quench on a
		theoretical level.
	\item[\Cref{chp:ml}] gives an overview of machine learning and the various models used
		within the project on a theoretical level.
	\item[\Cref{chp:problem}] gives some background for the original experiment and explains the
		structure of the dataset used throughout testing.
	\item[\Cref{chp:qrp}] introduces the Quench Recognition Problem, as well as the models
		utilized to solve the problem and the results obtained, alongside some
		considerations on the models' structure.
	\item[\Cref{chp:qlp}] introduces the Quench Localization Problem, the extension of previously
		obtained models to this new instance, and the final results.
	\item[\Cref{chp:conclusion}] among some final remarks, we introduce some ideas to further extend
		the project.
\end{description}
