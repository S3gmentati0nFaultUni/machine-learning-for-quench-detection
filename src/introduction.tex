\chapter{introduction}
This thesis originated as a cooperation between the departments of Computer Science and Physics of
Milan University in an attempt of confirming, the analytical method originally found by
\textsc{infn} and the University of Milan superconducting research group for the localization of
quench events in the High Order Corrector superconducting magnets \cite{mariotto2022}
\cite{mariotto2022-generic} that will be mounted in the \textsc{lhc} collider machine for thr High
Luminosity upgrade \cite{rossi2024} (Hi-Lumi for short).

The aim of the original work was, beginning from the harmonic decomposition of the magnetic
field produced by the superconducting magnet after a \emph{quench} event, which is a transition of the
material from the superconducting state to the normal conducting state, to find an analytical
explanation for the reconstructed harmonic content and correlate the information with the
localization of the quench superconducting coil in the magnet assembly. This problem will be
referred to, in this thesis, as the Quench Localization Problem (\qlp\ for short).

The aim of this thesis is to identify a model that can reliably reply to the question 'If quench
happened during the test run, which coil within the magnet did quench?' knowing the harmonic
decomposition of the magnetic field measured at the center of the magnet, where the beam will move
in production, this magnetic field is the result of the contribution of the magnetization of the
superconducting coils and the iron yoke that locks the coils in place. The analyses have been
conducted after the quench event manifested itself, to avoid damaging the material the magnet was
tested in a protected environment connected to a Quench Protection System to avoid material damage
(the experiment will be introduced and explained in \Cref{chp:problem}).

\smallskip

In the following I will be explaining some of the studies and results obtained in the field of
quench event recognition and quench event localization by using machine learning models to abstract
patterns from the measurements of magnetic field harmonics captured by the co-advisor for this
thesis, researcher Samuele, during his own experiments. The resulting models could be of use in the field of magnet testing and behavior explanation for future research work in the field of magnet manufacturing.

\smallskip

In this document I chose to approach the matter as follows:
\begin{description}
	\item[\Cref{chp:soupcond-quench}] gives an overview of superconductivity and quench on a
		theoretical level.
	\item[\Cref{chp:ml}] gives an overview of machine learning and the various models used
		within the project on a theoretical level.
	\item[\Cref{chp:problem}] explains the challenges associated with the project, the datasets
		that have been used and the various testing and model selection procedures.
	\item[\Cref{chp:qrp}] introduces the Quench Recognition Problem, as well as the models
		utilized to solve the problem and the results obtained, alongside some
		considerations on the models' structure.
	\item[\Cref{chp:qlp}] introduces the Quench Localization Problem, the extension of previously
		obtained models to this new instance, and the final results.
	\item[\Cref{chp:future}] lists other ideas that came along while designing a solution for
		the two problems at hand.
	\item[\Cref{chp:conclusion}] closes the dissertation with some on the design and development
		processes.
\end{description}
