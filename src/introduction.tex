\chapter{introduction}
This thesis was born as a cooperation between the department of Computer Science and Physics of
Milan University in an attempt of confirming or giving us a good reason to doubt the analytical
method originally found by Samuele Mariotto in his research on the localization of quench events for
the High Order Correctors superconducting magnets that will be mounted on the LHC experiment for the
HiLumi upgrade.

In the following I will be explaining some of the studies and results obtained in the field of
quench event recognition and quench event localization by using machine learning models to abstract
patterns from the measurements of magnetic field harmonics captured by Mariotto during his own
experiments.

In this document I chose to approach the matter as follows:
\begin{description}
	\item[\Cref{chp:soupcond-quench}] gives an overview of superconductivity and quench on a
		theoretical level.
	\item[\Cref{chp:ml}] gives an overview of the various models that have been used in the
		project on a theoretical level.
	\item[\Cref{chp:problem}] explains the challenges associated with the project, the datasets
		that have been used and the various testing and model selection procedures.
	\item[\Cref{chp:qrp}] introduces the Quench Recognition Problem, as well as the models
		utilized to solve the problem and the results obtained.
	\item[\Cref{chp:qlp}] introduces the Qench Localization Problem, the extension of previously
		obtained models to this new instance, and the final results.
	\item[\Cref{chp:future}] lists other ideas that came along while designing a solution for
		the two problems at hand.
	\item[\Cref{chp:conclusion}] closes the document with a couple of thoughts on the design and
		development process.
\end{description}
