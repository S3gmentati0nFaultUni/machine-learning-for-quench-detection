\chapter{Conclusions}
\label{chp:conclusion}
In this thesis we have proposed a solution to the problem of identifying quench events and
localizing them within the magnet structure using the harmonic decomposition of magnetic field. We
have provided a very strong solution for the first and we have provided a good starting point for
the second, during the project different ideas came to mind to branch the current solution and
further explore the interesting realm of machine learning applied to superconductor physics.

We remark that, while the analysis we have done has been thorough, the performance achieved and the
identified models only works on the specific case of the High order corrector quadrupolar magnets,
we have serious doubts that this analysis could be extended to any other quadrupole and we are sure
that extending it to magnets with a different number of poles would yield unusable results.

Lastly, while in the case of \qlp\ we could probably achieve higher performance by doing a more
thorough study of models built for the coils that are struggling the most, achieving higher
performance in both cases would require more data, but an increase in dataset size might leade to a
decrease in result interpretability.

\subsubsection{Analysis of the decision rules and general detection model}
In the project we have identified highly explainable models, capable of solving \qrp\ and \qlp\
maintaining high performance, an obvious next step is to open the models the models and abstract
useful rules capable of explaining quench and localizing events in certain coils. These rules could
be used:
\begin{itemize}
	\item In the field of sensor calibration, to isolate the harmonics that we truly need and
	      simplifying sensor design, while also making the Quench Protection System more efficient,
	\item Support in the magnet design process
\end{itemize}

\subsubsection{Fuzzification of quench-event description}
For both our problems the quench-event was described by a binary value, if $0$ then the sample
remained in the superconducting state, if $1$, then the material transitioned to the normal-conducting state. An option that came to mind was to change the description of quenches using variables capable of explaining a transition using a real number in the interval $[0, 1]$.

Using this description would be beneficial because a quench event could be described more thoroughly
by highlighting the rapidity of the transition, which could be used as a measure of its
disruptiveness, quench that spreads rapidly poses a larger threat than a quench that spreads very
slowly and therefore has a very high chance of disappearing without the intervention of the
Quench Protection System.

\subsubsection{High performance quench localization}
We tried to find explainable models capable of reliably detect and localize quench inside
quadrupoles, a different area of research consists in exploring extremely high performance models
for online quench detection and localization~\cite{hoang2021}~\cite{zhou2021}~\cite{einstein2023}.

Many are the alternatives: the \svcs\ explored in this thesis, neural networks or anomaly detection
via autoencoders. Clearly, pursuing such path forces us to invalid the explainability property we
granted throughout the project, unless explainable-ai models were used.

\subsubsection{Clustering-based preprocessing for classification}
Another technique that could be worth exploring is Clustering, we saw in the last chapter how the
points were very nicely distributed in bidimensional space, specifically for attribute \an, but also
for attribute \cnmod\ (after some extra work). We didn't have enough time to fully explore the
possibility of using clustering as a preprocessing step to then train a \dt\ on data that have
already been partly classified.
