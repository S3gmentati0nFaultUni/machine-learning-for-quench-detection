\chapter{Superconductivity and quench: an overview}
In this chapter I will explain some of the very basic concepts linked to superconductors and
superconductor quench.
\section{Superconductivity in theory}
Electrical motion in metallic or insulating materials is a very well studied and documented
phenomenon. When the first evidences of superconductivity were found in 1911
\cite{invention-superconductivity} it soon was discovered that
the underlying theory was not actually able to explain the newfound properties.
Let's consider a cable of length $l$ and cross-section $A$ exerting a resistance $R$, the material
has an associated value of resistivity $\rho$ which can be computed as follows
\begin{equation}
	\label{eq:resistivity-cable}
	\rho = \frac{RA}{l} = \frac{m}{ne^2\tau}
\end{equation}
because of the second part of \ref{eq:resistivity-cable} resistivity can also be defined as in dependence of the mass of the electron $m$, the charge
density $n$, the electrical charge of the electron $e$ and the relaxation time of the electron, which indicates the time interval occurring between two successive collisions between
electrons inside a conductor, $\tau$.

The resistivity of any conductor can also be written in terms of the sample's temperature and a reference
value of resistivity ($\rho_0$) and temperature ($T_0$)
\begin{equation}
	\label{eq:resistivity-func-of-temp}
	\rho_T = \rho_0[1 + \alpha(T - T_0)]
\end{equation}
$\alpha$ is a regularization term known as the temperature coefficient of resistivity, which is the
ratio of resistance change to temperature change.

As it's explained in \cite{slimani2022superconducting} the result of fusing solid band theory with
\ref{eq:resistivity-func-of-temp} behavior of metals and insulators can be explained in terms
of the freedom of movement of electrons in the crystal. An increase in temperature for a metal
translates into an increase in resistivity while insulators experience a decrease in resistivity
and some start behaving like metals.
\section{Type I superconductors}
\section{Type II superconductors}
\section{Quench in theory}
