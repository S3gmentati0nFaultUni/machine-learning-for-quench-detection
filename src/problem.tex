\chapter{The problem}
\label{chp:problem}
The problem that we solved in this thesis is to provide the \textsc{infn} with a machine learning
capable of, given $15$ harmonics of magnetic field measured on a superconducting magnet:
\begin{enumerate}
	\item Recognize whether the magnet underwent a quench event during operation, we will refer
	      to it as Quench Recognition Problem (\qrp) from now on,
	\item Recognize, if the superconductor quenched, how many coils underwent the quench
	      transition and specifically which coils in the magnet transitioned; we will refer to
	      it as Quench Localization Problem (\qlp) from now on.
\end{enumerate}
A field harmonic is a complex function capable of descibing the contribution of a certain multipolar
field (e.g. the coefficient associated with the dipole is $C_0$, while $C_1$ is the quadrupolar
coefficient) to the entirety of the magnetic field, which is a vectorial and continuous quantity.

This problem, as was already said in previous chapters, is a natural extension of the work of
Samuele Mariotto \cite{mariotto2022}\cite{mariotto2022-generic}, who analyzed the behavior of
a gamma of magnets differing in the number of poles. The original dataset only contained the quench
events recorded for every type of magnet, since the number of events for every other magnet type was
less than the number of quenches recorded for the sole quadrupoles (a total of $177$), we chose to
aim for the creation of a model capable of identifying and locating quench events within a
quadrupolar structure.

As was shown in the theoretical introduction to machine learning (\Cref{chp:ml}) we were aiming to
find models capable of giving a very good explanation of the classification outcomes. Having the
ability of explaining what produced the event can be of great help in the context of magnet design,
quench explanation and instrument calibration (e.g. If a new quench antenna needs to be designed the
researchers know that some magnetic field harmonics can be ignored because they do not influence the
result).

In the following two sections we will explore the structure of the dataset used for the project and
then we will learn about the constraints imposed for reproducibility's sake.

\section{The datasets and their meaning}
The dataset used for the project contained a set of four different measurements, each result of a
previous manipulation done by researcher Mariotto on the original measurements of the magnetic field
harmonics.

\begin{itemize}
	\item \emph{An}: the imaginary part of the magnetic field harmonics, since the measurements
	      done for the original experiments were done in a \emph{skew}
	      configuration\comment{\footnote{Magnets are said to be mounted in skew configuration when the
		      positioning of the magnets is at an angle relative to each other, this improves
		      performance and efficiency}}{Quote someone to make sure that the statement holds for
	      accelerator physics. It's important o be able to understand this question: Why should we mount
	      magnets skewed?} this was the
	      table that supposedly contained more information;
	\item \emph{Bn}: the real part of the magnetic field harmonics;
	\item \emph{Cnmod}: the absolute value of the complex coefficient $C_n$;
	\item \emph{Phi}: the phase of the magnetic field harmonics, since in the original
	      analytical method the conclusions regarding the quenched coil within the maagnet were taken
	      based on this value the naïve hypothesis was that the \qlp solution would strongly depend on
	      this.
\end{itemize}

\section{Model selection and model testing procedures}
Reproducibility is an extremely important property of any experiment, as was said in the chapter
dedicated to a light introduction to machine learning theory (\Cref{chp:ml}), it's an important
first step to set the value of the random number generators.

The next step to enforce reproducibility is by use shared pipelines.

Every dataset used in the project is processed using a common pipeline that builds three different
dataframes that are serialized for later reuse:
\begin{itemize}
	\item The \emph{merged} dataframe contains all of the field harmonics for a certain table
	      (e.g. An) or mix of tables (e.g. An, Bn, Cnmod), berfore serialization the dataframe is
	      normalized using an instance of the StandardScaler class, contained in scikit-learn, trained
	      on the dataframe just created (minus the label columns).
	\item The \emph{safe} dataframe contains a small part of the overall data available to us
	      (29 samples), these samples were kept away until the experiments were considered complete.
	      This dataset allowed us to perform a blind test on the model to see whether it was able to
	      generalize well what it learned.
	\item The \emph{reduced} dataframe contains the rest of the points in the original table,
	      it is the main dataframe used to perform all of the tests and experiments.
\end{itemize}

When it came to model selection and model testing we chose to perform a Nested Cross Validation
procedure (\ncv). As was said in \Cref{chp:ml} the models used during the project were mostly
supervised, and one of the biggest issues with such models is overfitting. With the single $k$-fold Cross
Validation procedure the model is tested in such a way that overfitting is not an issue anymore.
