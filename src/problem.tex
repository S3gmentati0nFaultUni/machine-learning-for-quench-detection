\chapter{The problem}
\label{chp:problem}
Given $15$ harmonics of magnetic field measured on a superconducting magnet, during a test run, our
objective is to provide the \textsc{infn} with a machine learning model capable of:
\begin{enumerate}
	\item Recognizing whether the magnet underwent a quench event during operation, we will refer
	      to it as Quench Recognition Problem (\qrp) from now on, and will be treated in
	      \Cref{chp:qrp},
	\item Recognize, if the superconductor quenched, how many coils underwent the quench
	      transition and specifically which coils in the magnet transitioned; we will refer to
	      it as Quench Localization Problem (\qlp) from now on and it will be treated in
	      \Cref{chp:qlp}.
\end{enumerate}
A field harmonic is a complex function capable of descibing the contribution of a certain multipolar
field (e.g. the coefficient associated with the dipole is $C_0$, while $C_1$ is the quadrupolar
coefficient) to the entirety of the magnetic field, which is a vectorial and continuous quantity.

\medskip

This problem, as was already said in previous chapters, is a natural extension of the work of
Samuele Mariotto \cite{mariotto2022}\cite{mariotto2022-generic}, who analyzed the behavior of
a gamma of magnets differing in the number of poles. The original dataset contained the quench
events recorded for every type of magnet, since the number of events for the sole
quadrupoles\footnote{Due to some testing problems in the original work, two different quadrupoles
	were tested (1 and 1b) in the following} made up almost the entirety of the dataset we chose to work
exclusively on them.

\medskip

Due to the nature of the problem, we were looking for models that could solve both our problems
while keeping a very high level of explainability. Being able to explain which magnetic field
harmonics are more important to identify reliably a quench event can be of great help in the context of magnet design,
quench explanation and instrument calibration (e.g. \comment{If a new quench antenna needs to be designed the
researchers know that some magnetic field harmonics can be ignored because they do not influence the
result}{Check with Samuele if this makes sense}).

\medskip

In the following two sections we will explore the structure of the dataset used for the project and
then we will learn about the constraints imposed for reproducibility's sake.

\section{The datasets and their meaning}
The original experiment measured magnetic field harmonics, since they are complex numbers they can
be manipulated to obtain different values (e.g., imaginary part, real part or phase of the complex
number). The dataset used for the project contained a set of four different measurements, each result of a
previous manipulation, done by researcher Mariotto, on the original measurements. The meaning of
each table is explained briefly here below:
\begin{itemize}
	\item \emph{An}: the imaginary part of the magnetic field harmonics, since the magnets
	      during the test run were mounted in \emph{skew} configuration\comment{\footnote{Magnets are said to be mounted in skew configuration when the
		      positioning of the magnets is at an angle relative to each other, this improves
		      performance and efficiency}}{Quote someone to make sure that the statement holds for
	      accelerator physics. It's important o be able to understand this question: Why should we mount
	      magnets skewed?} this table was the one that supposedly provided the more information;
	\item \emph{Bn}: the real part of the magnetic field harmonics, for the same reason above
	      not as much performance was expected from this table;
	\item \emph{Cnmod}: the absolute value of the complex coefficient $C_n$, it's the
	      combination of An and Bn tables, and it's also the table that suffers more the
	      effect of the centering (for measurement purposes), we expected that this table
	      would perform very well in the \qrp task;
	\item \emph{Phi}: the phase of the magnetic field harmonics, since in the original
	      analytical method the conclusions regarding the quenched coil within the magnet were taken
	      based on this value the naïve hypothesis was that the \qlp solution would strongly depend on
	      this.
\end{itemize}

\section{Model selection and model testing procedures}
Reproducibility is an extremely important property of any experiment, to cover the basics all the
seeds for random number generators were set to the same value. The next step to enforce
reproducibility is by using shared pipelines.

Every dataset used in the project is processed using a common pipeline that builds three different
dataframes that are serialized for later reuse:
\begin{itemize}
	\item The \emph{merged} dataframe contains all the field harmonics for a certain table
	      (e.g. An) or mix of tables (e.g., An, Bn, Cnmod), before serialization the dataframe is
	      normalized using an instance of the StandardScaler class, contained in scikit-learn, trained
	      on the dataframe just created (minus the label column(s)).
	\item The \emph{safe} dataframe contains a small part of the overall data available to us
	      (29 samples), these samples were kept away until the experiments were considered complete.
	      This dataset allowed us to perform a blind test on the best models to see whether they
	      were able to generalize what they previously learned.
	\item The \emph{reduced} dataframe contains the rest of the points in the original table,
	      it is the main dataframe used to perform all tests and experiments.
\end{itemize}

When it came to model selection and model testing we chose to perform a Nested Cross Validation
procedure ($\ncv$). This technique comes with two incredible advantages that suit this problem very
well:
\begin{enumerate}
	\item $\ncv$ is suitable for situations in which the data is not abundant
	      \cite{Larracy2021} due to the heavy reuse of data during the model selection
	      process, giving also a less biased performance estimation (due to the averaging of
	      the results on more folds),
	\item $\ncv$ is more resilient against overfitting thanks to the 'physical' separation
	      between the hyperparameter selection and testing procedures.
\end{enumerate}
In the following section we will give a brief introduction of the Cross validation procedure.

\subsubsection{Cross Validation}
In \Cref{chp:ml} we talked about the simple train-test splitting procedure, now we will introduce an
alternative technique for splitting that gives us a powerful tool to prevent overfitting and get
less biased performance measures while also doing hyperparameter selection.

\medskip

In general, given a dataset $D$, $k$-fold $\cv$ \cite{ZhouZhi-Hua2021ML} is a technique for splitting a dataset in $k$
different folds $\fold{i}$, the folds are non-overlapping and about the same size, therefore
$\fold{i} \cap \fold{j} = \emptyset \hspace{5pt} \forall i, j \in k$, and the union of all the folds
is the original dataset ($\bigcup_{i \in \{1, \ldots, k\}} \fold{i} = D$).

Let us consider the trivial train-test environment, if we used the procedure introduced in
\Cref{chp:ml}, we would split the original dataset into the training set $T$ and the testing set
$G$, then train a model $\model$ on $T$ and check the generalization performance on $G$. Once the test is completed we have the performance of the model.

A series of questions normally arise whenever we think about this procedure, we will list them below
and provide an answer.

\medskip

'What if the $T$-$G$ split chosen was the lucky one and the performance we obtain out of the last
reading are overly optimistic?'

\smallskip

This is where $\cv$ comes in, to reduce the probability that the performance of the model
are good just because of a statistical anomaly. If we have a dataset of $250$ samples, by doing a
$5$ fold $\cv$, $5$ different folds of $50$ samples are generated, each fold will be used once to
test the generalization performance of the model trained on the remaining $200$ samples.

\medskip

'What value of $k$ should be chosen?'

\smallskip

The number of folds to be used for $\cv$ becomes another hyperparameter of the problem, since it
depends on many factors like the amount of data available to us, a very big value of $k$ will
decrease the size of the testing or generalization fold, which reduces the trust that we can put in
the generalization performance obtained for the model; too small a value of $k$ makes the $\cv$ more
robust but makes it at the same time computationally expensive.

Leave One Out (\textsc{loo-cv}) is an extreme form of $\cv$ which divides $D$ into folds containing
only one element, per what was said earlier $n$ different procedures of training and testing will be
carried out, giving very reliable results, at the expense of a high computational load. Due to the length and complexity this \textsc{loo-cv} is rarely used in real experiments (despite the high grade of reliability of the results), an efficient version of \textsc{loo-cv} is discussed in \cite{shao2016}.

\medskip

To introduce Nested $\cv$ ($\ncv$) we will consider the case of having to choose the best model,
training it and then testing its performance. In the base case the dataset $D$ can be tripartitioned
as follows:
\begin{itemize}
	\item \emph{Training set} $T$,
	\item \emph{Validation set} $V$, this set of samples will be used to test the performance of the
	      model found after the training step,
	\item \emph{Generalization set} $G$, after the model is retrained on $T + V$ the performance are
	      tested on $G$ to see if the model is capable of generalizing effectively.
\end{itemize}

If we apply this simple procedure to select the model, train it and then test it; how can we
guarantee that the model chosen is the best one, or at least that it's good enough?

\smallskip

Quite simply, we can't give any guarantees, we might have gotten the worst possible mix of
hyperparameters and still get good performance during the validation and generalization tests due to
statistical anomalies.
\begin{figure}
	\centering
	\includegraphics[scale=.3]{./img/nested-cv.png}
	\label{fig:nested-cv}
	\caption{Visualization of the nested cross validation procedure taken from \cite{pain2020}}
\end{figure}
\comment{If I want to add a picture for simple cross validation then something ad hoc should be
created so that the style is kind of the same}

Compared to the method that we just talked about $\ncv$, which is sketched in $\Cref{fig:nested-cv}$, consists in two simple $\cv$
procedures nested, The outer one is used for testing the generalization performance of the model,
the internal one is used to do model selection and is constructed on the training set used in the
outer loop. Repeating the model selection process gives us better hyperparameter selection and
lowers the chances of us picking a model that performs good only on paper, it also gives us stronger
performance metrics for the best model coming out of the inner fold due to the averaging of the
metrics over different folds.

\medskip

In the project, we chose to apply a $5 \times 5$ $\ncv$ procedure, which means that, out of
the $250$ samples that form the original dataset $D$:
\begin{itemize}
	\item The data is first split into $5$ folds of $50$ samples. One fold of $50$ samples will
	      be kept on the side for the generalization test, the remaining $200$ samples will be used to do both training and model selection;
	\item The training set is now split again in $5$ folds, $160$ samples will be used to do
	      model selection (therefore hyperparameter space search), $40$ samples will be used
	      to validate the performance that has been found;
	\item The model found at the previous step is retrained on the whole $200$ training samples and then
	      the performance are tested on the fold that was kept aside in the first step.
\end{itemize}
If we consider once again $\Cref{fig:nested-cv}$ it's clear that what comes out of the $\ncv$
procedure are $5$ different models, which one should be considered the best model overall? In
practice the model that is kept after the $\ncv$ is the one that has the lowest error overall.

The scikit-learn library contains many $\cv$ implementations that can give different or stronger
guarantees compared to the $\cv$ procedure shown in this section, the splitting procedure is also
handled differently based on the type of KFold object used.

The closing section for this chapter will contain a summarization of the experiment's
characteristics.
\subsection{Experimental setup}
The project was developed using the latest version of the Python programming language
(\texttt{3.10.12} at the moment of writing), a virtual environment was set up to, favor ease of use,
and the libraries were handled using the pip package manager (also in the last version, which is the
\texttt{24.2}).

\medskip

We handled datasets by using a mix of Pandas (version $2.2.3$ at the moment of writing) DataFrames
and Numpy (version $2.1.2$ at the moment of writing) $n$-dimensional arrays. We took most functions
for preprocessing and machine learning models from the scikit-learn (version $1.5.2$ at the moment
of writing) library.

\medskip

The seeds from the various random number generators have been set to the common value of
\href{https://www.google.com/search?q=the+answer+to+life+the+universe+and+everything&num=10&client=firefox-b-d&sca_esv=a81abf9bb67ffd9b&sxsrf=AHTn8zo6RKep_zuEvIhJb5nuAGh5xAERLg\%3A1739739141586&ei=BVCyZ_C9I9mLi-gP7e-B8Ac&oq=the+answer+to+&gs_lp=Egxnd3Mtd2l6LXNlcnAiDnRoZSBhbnN3ZXIgdG8gKgIIADIIEAAYgAQYywEyCBAAGIAEGMsBMgUQABiABDIIEAAYgAQYywEyCBAAGIAEGMsBMggQABiABBjLATIIEAAYgAQYywEyCBAAGIAEGMsBMggQABiABBjLATIIEAAYgAQYywFIjCJQqQlYiRlwA3gBkAEAmAFuoAHICaoBBDExLjO4AQPIAQD4AQGYAhGgAv0JwgIKEAAYsAMY1gQYR8ICChAjGIAEGCcYigXCAgsQABiABBixAxiDAcICCxAuGIAEGLEDGIMBwgIOEC4YgAQYsQMY0QMYxwHCAg4QLhiABBixAxiDARiKBcICERAuGIAEGLEDGNEDGIMBGMcBwgIMECMYgAQYExgnGIoFwgIEECMYJ8ICDRAuGIAEGEMY1AIYigXCAgoQABiABBhDGIoFwgIOEC4YgAQYxwEYjgUYrwHCAgoQLhiABBhDGIoFwgIIEC4YgAQYsQPCAggQABiABBixA8ICCxAuGIAEGLEDGNQCwgIFEC4YgATCAggQLhiABBjLAcICCxAuGIAEGNEDGMcBmAMAiAYBkAYIkgcEMTIuNaAHrboC&sclient=gws-wiz-serp}{42} to prevent too much deviation in the experiments.

\medskip

The original contained $279$ samples, for \qrp, each sample consisted of $15$ harmonics and a Label
stating whether the sample represents a quench event or not, for \qlp, each sample had $4$ different
labels associated to it representing whether one of the coils ($0$ for East, then: North, West and
South) quenched or not.

When the dataset is divided between reduced and safe dataframes the distribution of the labels is
preserved by using stratified sampling, in the case of multiclass-classification was required I decided to use the stratified sampling technique provided in the scikit-multilearn library (\cite{skmlearn}).

\medskip

As far as the model selection, training and testing pipeline is concerned, as was said in the
previous section, we chose to use a $5\times5$ $\ncv$ procedure, the best model chosen was the one
contained in the \_best\_estimator attribute of the GridSearchCV procedure, the resulting
model is then trained on the whole reduced dataset and serialized using pickle.

\medskip

Experiments were conducted on three different computers running different architectures and
different operating systems, but the results did not change due to the standard-oriented approach,
\Cref{tbl:computers} for a summary of the computers we used.
\begin{center}
	\begin{longtable}{|c|c|c|c|}
		\caption{System information}\label{tbl:computers}
		\\\textbf{}                             & \textbf{Pigna}
		                                                & \textbf{Mattone}
		                                                & \textbf{Topone}
		\\\hline\hline
		\endfirsthead\hline\endlastfoot

		\textbf{CPU manufacturer}                       & Intel
		                                                & AMD
		                                                & Intel
		\\\hline
		\textbf{CPU model}                              & i$5$ $5287\textsc{u}$
		                                                & Ryzen $3700\textsc{x}$
		                                                & i$7$ $4770$
		\\\hline
		\textbf{CPU core count}                         & $2\textsc{c}$ $2\textsc{t}$
		                                                & $8\textsc{c}$ $16\textsc{t}$
		                                                & $4\textsc{c}$ $8\textsc{t}$
		\\\hline
		\textbf{Litography (nm)}                        & $14$
		                                                & $7$
		                                                & $22$
		\\\hline
		\textbf{Launch Date}                            & Q$1$ 2015
		                                                & Q$2$ 2019
		                                                & Q$2$ 2013
		\\\hline
		\textbf{Base frequency (GHz)}                   & $2.90$
		                                                & $4.125$
		                                                & $3.40$
		\\\hline
		\textbf{Boost frequency (GHz)}                  & $3.30$
		                                                & $4.40$
		                                                & $3.90$
		\\\hline
		\textbf{Cache} ($\textsc{mb}$)                  & $3$
		                                                & L1: $0.512$, L2: $4$, L3:$32$
		                                                & $8$
		\\\hline
		\textbf{TDP\footnote{Thermal Design Power} (W)} & $28$
		                                                & $84$
		                                                & $65$
		\\\hline
		\textbf{Memory size} ($\textsc{gb}$)            & $8$
		                                                & $16$
		                                                & $16$
		\\\hline
		\textbf{Operating System}                       & Pop\!\_OS $22.04$ LTS
		                                                & Pop\!\_OS $22.04$ LTS
		                                                & Fedora $39$
		\\\hline
	\end{longtable}
\end{center}








