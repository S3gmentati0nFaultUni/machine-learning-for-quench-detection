\chapter{Further developments and ideas}
\label{chp:future}
